\chapter*{Abstract}
\phantomsection
\addcontentsline{toc}{chapter}{Abstract}

\small


\paragraph*{Abstract:}  \lipsum[1] \\
\textbf{Keyword:}  Energy Monitoring, Fuel Cell, Optimization, Energy management, Fuzzy Logic, fuel-cell hybrid electric vehicles.

 % Start a group
	\begin{Arabic}
		\paragraph*{ملخص:}
الانتقال نحو تقنيات النقل الأكثر نظافة وكفاءة أدى إلى الاهتمام المتزايد بمركبات الهيدروجين الهجينة   كبديل واعد للمحركات التقليدية ذات الاحتراق الداخلي. ومع ذلك، يبقى إدارة تدفق الطاقة بين خلايا الوقود والبطاريات تحدياً كبيرً. تركز هذه الأطروحة على تطوير وتحسين نظام إدارة الطاقة المعتمد على المنطق الضبابي لتحسين كفاءة مركبات الهيدروجين الهجينة من ناحية اقتصاد الوقود والأداء في ظل ظروف القيادة المختلفة. تبدأ هذه الدراسة بدراسة أنظمة إدارة الطاقة المعتمدة على المنطق الضبابي وقدرته على التعامل مع احتياجات الطاقة غير الخطية والديناميكية لمركبات الهيدروجين الهجينة، مع تسليط الضوء على مزايا المنطق الضبابي مثل سهولته وقابليته للتكيف، مقارنةً باستراتيجيات التحكم التقليدية.

\textbf{الكلمات المفتاحية:} مراقبة الطاقة، خلية وقود، تحسين، إدارة الطاقة، المنطق الضبابي، المركبات الكهربائية الهجينة التي تعمل بخلايا الوقود 
	\end{Arabic}


\begin{french}
\paragraph*{Résumé:} \lipsum[1].\\
	\textbf{Mots-clés: } Surveillance de l'énergie, Pile à combustible, Optimisation, Gestion de l'énergie, Logique floue, Véhicules hybrides électriques à pile à combustible 
	
\end{french}

\newpage

