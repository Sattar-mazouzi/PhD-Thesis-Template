\chapter*{Abstract}
\phantomsection
\addcontentsline{toc}{chapter}{Abstract}

\small


\paragraph*{Abstract:}  \lipsum[1] \\
\textbf{Keyword:}  Energy Monitoring, Fuel Cell, Optimization, Energy management, Fuzzy Logic, fuel-cell hybrid electric vehicles.

 % Start a group
	\begin{Arabic}
		\paragraph*{ملخص:}
الانتقال نحو تقنيات النقل الأكثر نظافة وكفاءة أدى إلى الاهتمام المتزايد بمركبات الهيدروجين الهجينة   كبديل واعد للمحركات التقليدية ذات الاحتراق الداخلي. ومع ذلك، يبقى إدارة تدفق الطاقة بين خلايا الوقود والبطاريات تحدياً كبيرً. تركز هذه الأطروحة على تطوير وتحسين نظام إدارة الطاقة المعتمد على المنطق الضبابي لتحسين كفاءة مركبات الهيدروجين الهجينة من ناحية اقتصاد الوقود والأداء في ظل ظروف القيادة المختلفة. تبدأ هذه الدراسة بدراسة أنظمة إدارة الطاقة المعتمدة على المنطق الضبابي وقدرته على التعامل مع احتياجات الطاقة غير الخطية والديناميكية لمركبات الهيدروجين الهجينة، مع تسليط الضوء على مزايا المنطق الضبابي مثل سهولته وقابليته للتكيف، مقارنةً باستراتيجيات التحكم التقليدية. تم استخدام منهج لتحسين متغيرات المتحكم الضبابي، بما في ذلك خصائص دوال الانتماء وأوزان القواعد، ومجالات المتغيرات الضبابية، باستعمال خوارزميات التحسين المتقدمة.  في البداية، تم تطبيق خوارزمية تحسين سرب الجسيمات لضبط خصائص دوال الانتماء في نظام إدارة الطاقة الضبابي في ظروف قيادة مختلفة. تم استخدام متحكمين بدوال انتماء مختلفة، النظام الأول يعتمد على دوال شبه منحرف والنظام الثاني يعتمد على دوال مثلثية. أظهرت النتائج تحسنًا ملحوظًا في اقتصاد الوقود والتحكم في مستوى شحن البطارية مقارنة بأنظمة إدارة الطاقة غير المحسّنة وتلك المبنية في نظام المحاكات، ومع ذلك، لا يزال هناك مجموعة من المتغيرات الضبابية التي تؤثر على أداء النظام ويجب تحسينها، مما يبرز الحاجة إلى حل أكثر تكيفًا. للتغلب على هذه القيود، تم تطوير إطار عمل شامل لتحسين متغيرات نظام إدارة الطاقة المبني على المنطق الضبابي باستخدام خوارزميات الجينات لضبط متغيرات المنطق الضبابي المتعددة. تم تنفيذ عملية التحسين على أربع مراحل متميزة، حيث ركزت كل مرحلة على مجموعة متغيرات مختلفة. أدى هذا النهج المنظم إلى تحسينات ملحوظة في كفاءة النظام، واقتصاد الوقود، وتوازن توزيع الطاقة في ظروف قيادة مختلفة. أكدت الدراسات المقارنة والتحليلات البصرية على تفوق نظام إدارة الطاقة المحسن باستخدام خوارزمية الجينات مقارنة بأنظمة إدارة الطاقة التقليدية وغير المحسّنة. كما تبرز الأطروحة الصياغة الرياضية لمشكلة التحسين، مع تفاصيل متغيرات القرار، ومساحات البحث، والدوال الهدفية، والقيود المصممة لتحسين استهلاك الوقود مع تقليل تشغيل خلايا الوقود في المناطق منخفضة الكفاءة. من القيود الرئيسية، الحفاظ على حالة شحن البطارية ضمن الحدود المقبولة. تختتم الدراسة بإظهار إمكانيات خوارزميات الجينات وسرب الجسيمات لتحسين فعالية أنظمة إدارة الطاقة المعتمدة على المنطق الضبابي لمركبات الهيدروجين الهجينة. لذالك، ستركز الأبحاث المستقبلية على استكشاف استراتيجيات إدارة طاقة أكثر تكيفًا يمكنها التكيف ديناميكيًا مع ظروف القيادة المختلفة في الوقت الفعلي. بالإضافة إلى ذلك، من المهم تحسين مزيد من متغيرات المنطق الضبابي. كما يُوصى أيضًا بإدراج المتغيرات المختلفة لنموذج المركبة، مثل درجة حرارة البطارية، وسرعة المحرك، وتضاريس الطريق، لإنشاء نظام إدارة طاقة أكثر قوة ودقة. ستساهم هذه الدراسة بشكل كبير في تحسين استراتيجيات إدارة الطاقة لمركبات الهيدروجين الهجينة وتوفير رؤى قيمة لتصميم أنظمة تحكم ذكية وفعالة من حيث إدارة الطاقة مما سيساهم في تطوير تقنيات نقل مستدامة وموثوقة. \\
\textbf{الكلمات المفتاحية:} مراقبة الطاقة، خلية وقود، تحسين، إدارة الطاقة، المنطق الضبابي، المركبات الكهربائية الهجينة التي تعمل بخلايا الوقود 
	\end{Arabic}


\begin{french}
\paragraph*{Résumé:} \lipsum[1].\\
	\textbf{Mots-clés: } Surveillance de l'énergie, Pile à combustible, Optimisation, Gestion de l'énergie, Logique floue, Véhicules hybrides électriques à pile à combustible 
	
\end{french}

\newpage

