\chapter{Methods and Experimental Setup} \label{CH2}

\section{Introduction }
\lipsum[1-3]
\subsection{Background Concepts}
\lipsum[1]
\begin{table}[h!]
	\centering
	\caption{Example of a numeric table with alignment.}
	\small
	\begin{tabular}{|l|r|r|r|}
		\hline
		\textbf{Method} & \textbf{Case 1} & \textbf{Case 2} & \textbf{Case 3} \\ \hline
		Approach A & 12.5 & 15.0 & 18.3 \\ \hline
		Approach B & 10.2 & 14.8 & 17.9 \\ \hline
		Approach C & 11.0 & 13.5 & 19.1 \\ \hline
	\end{tabular}
	\label{tab:numeric}
\end{table}
	\subsubsection{Basic Definitions}
	\subsubsection{Historical Context}

\subsection{Motivation and Scope}
	\subsubsection{Problem Statement}
	\subsubsection{Research Objectives}

\section{Methodology Examples}
\lipsum[1]
	\subsection{Experimental Setup}
		\subsubsection{Simulation Environment}
		\subsubsection{Hardware Configuration}

\section{Results and Discussions}
\lipsum[1]
	\subsection{Numerical Results}
	\lipsum[1]
		\subsubsection{Case Study 1}
		\lipsum[1]
		\subsubsection{Case Study 2}
		\lipsum[1]

\subsection{Discussion of Findings}
\lipsum[1]
	\subsubsection{Comparison with Literature}
	\lipsum[1]
	\subsubsection{Limitations of the Study}
	\lipsum[1]

\section{Conclusion }
\lipsum[1-3]
